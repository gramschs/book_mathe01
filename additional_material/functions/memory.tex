\RequirePackage[l2tabu, orthodox, abort]{nag}
\documentclass[paper=a4, fontsize=11pt, parskip=half]{scrartcl}
\usepackage[a4paper, width=170mm, height=250mm]{geometry}

\usepackage[usenames]{color} %used for font color
\usepackage{amssymb} %maths
\usepackage{amsmath} %maths
\usepackage[no-math]{fontspec}
\usepackage{unicode-math}
\usepackage{libertinus}
\usepackage{tikz}

% Keine Seitennummerierung
\pagestyle{empty}

% Befehl für ein Memory-Kartenpaar
\newcommand{\memorypaar}[2]{%
    \begin{tikzpicture}[baseline=(current bounding box.center)]
        % Formel-Karte (links)
        \draw[thick] (0,0) rectangle (8,8);
        \node at (4,4) {\Large #2};
        
        % Bild-Karte (rechts) - 2cm Abstand zwischen den Karten
        \draw[thick] (9,0) rectangle (17,8);
        \node at (13,4) {\includegraphics[width=7.8cm,height=7.8cm,keepaspectratio]{#1}};
    \end{tikzpicture}%
}

% Befehl für ein Memory-Kartenpaar mit mehrzeiliger mathematischer Formel
\newcommand{\memorypaargather}[2]{%
    \begin{tikzpicture}[baseline=(current bounding box.center)]
        % Formel-Karte (links)
        \draw[thick] (0,0) rectangle (8,8);
        \node at (4,4) {\Large $\begin{gathered} #2 \end{gathered}$};
        
        % Bild-Karte (rechts) - 2cm Abstand zwischen den Karten
        \draw[thick] (9,0) rectangle (17,8);
        \node at (13,4) {\includegraphics[width=7.8cm,height=7.8cm,keepaspectratio]{#1}};
    \end{tikzpicture}%
}


\begin{document}
\centering

\memorypaar{function01_diagonal.pdf}{f(x) = x}

\memorypaar{function02_parabola.pdf}{$f(x) = x^2$}

\memorypaar{function03_cubic.pdf}{$f(x)=x^3$}

\memorypaar{function04_4th.pdf}{$f(x)=x^4$}

\memorypaar{function05_negative_exponent_1.pdf}{$f(x)=x^{-1}=\frac{1}{x}$}

\memorypaar{function06_negative_exponent_2.pdf}{$f(x)=x^{-2}=\frac{1}{x^2}$}

\memorypaar{function07_sqrt.pdf}{$f(x)=\sqrt{x}$}

\memorypaar{function08_exp.pdf}{$f(x)=e^{x}=\exp(x)$}

\memorypaar{function09_log.pdf}{$f(x)=\ln(x)$}

\memorypaar{function10_abs.pdf}{$f(x)=|x|$}

\memorypaar{function11_sin.pdf}{$f(x)=\sin(x)$}

\memorypaar{function12_cos.pdf}{$f(x)=\cos(x$)}

\memorypaargather{function13_tan.pdf}{f(x)=\tan \\ \text{D} = \mathbb{R} \setminus \{ k\pi+\frac{\pi}{2} \,\large|\, k\in\mathbb{Z}\}}

\memorypaar{function14_heaviside.pdf}{$H(x) = \begin{cases} 
0 & \text{für } x < 0 \\
0.5 & \text{für } x = 0 \\
1 & \text{für } x > 0
\end{cases}$}

\memorypaar{function15_gauss.pdf}{$f(x)=\frac{1}{\sqrt{2\pi}}e^{-\frac{1}{2}x^2}$}


\end{document}